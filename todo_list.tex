\documentclass[8pt]{extarticle}

\usepackage[a4paper, total={17cm, 25cm}]{geometry}
\usepackage{outlines}
\usepackage{soul}
\usepackage{hyperref}
%-------------------------------------------------------------------------------
%---------------------- The 80-char/line limit: easier to read and track changes
\begin{document}




\begin{outline}[]
.\\
\hrulefill 
\0 {\bf {\large General Comments To Discuss}}
    %\1 \st{Cross comment once done.}
    \1 State the tight connection between SIS and LWE.
    \1 Another thing: the canonical embedding in section 7.2 is usually used to define the norm of a polynomial in the context of Ring SIS, but we don’t talk about using that for norm in section 7.2. Is that not needed in Ring LWE?
    \1 The whole section 7 is pretty hard going (for me), because it’s so abstract. In a way, if a reader doesn’t care about the security proofs of Ring LWE (to begin with), section 7 can be skipped (on a first reading).

  
\hrulefill 
\0 {\bf {\large Kelvin}}
    \1 Finish section 9 on HE.
    \1 Add a table of notations. 
    
    
%check security parameter symbol, make consistent if possible and link to the explain in intro to crypto. 
%RLWE:
%security par is a power of 2. 
%rlwe hardness is based on search svp, unlike in general lattices, in ideal lattices the decision svp is easy (lemma 2.9 lyub). 
%lyub sec1.1 has a semantically secure encryp scheme.
%sec1.2 raised a question about the search->decision, similar to what Kee Siong has asked. 
%for average-case decision rlwe, the n para of gaussian are kept secret def3.5
%a key benefit of working with canonical embedding is the fact that its axes will simply be permuted when the galois group acts on the prime ideals, see the paragraph above sec1.3
%

	


\hrulefill 
\0 {\bf {\large Kee Siong}}
    \1 Enter to do here. 
    
\hrulefill 
\0 {\bf {\large Mike}}
    \1 Enter to do here. 
   

\end{outline}
\end{document}